\documentclass[12pt, a4paper]{article}

\usepackage{amssymb, amsmath, amsthm, enumitem}
\usepackage{subfiles}
\usepackage[margin=2cm]{geometry}

\newtheorem{theorem}{Theorem}[section]
\theoremstyle{definition}
\newtheorem{definition}[theorem]{Definition}

\usepackage{dsfont, bbm, mathrsfs}

\usepackage[cmintegrals,cmbraces, varg]{newtxmath}
\usepackage{ebgaramond-maths}
\usepackage[T1]{fontenc}

\begin{document}
    \section{EB Garamond}
    \begin{definition}[Lebesgue Integral]
        Let $(X, \mathcal{M}, \mu)$ be a measure space. Let $f: X \to \mathbb{C}$ be a $\mathcal{B}(\mathbb{C})/\mathcal{M}$-measureable 
        function. Define the following formulations of an integral:
        \begin{itemize}
            \item If $f$ takes on only finitely many values in $\mathbb{R}^+$, then give the standard representation of $f$ 
            as the simple function:
            \[
                f(x) = \sum_{i} y_i \mathbbm{1}_{\{f(x) = y_i\}}(x).
            \]
            Then define the Lebesgue integral of $f$ with respect to $\mu$ as:
            \[
                \int_X f(x)\;d\mu(x) = \sum_{i} y_i \mu(\{f(x) = y_i\}).
            \]
            \item If $f$ takes on values in $\mathbb{R}^{+}$, then define the Lebesgue integral of $f$ with respect to $\mu$ 
            as:
            \[
                \int_X f(x)\;d\mu(x) = \sup\left\{\int_X g(x) \;d\mu(x)\mid g\le f \text{ and } g \text{ is a simple function}\right\}.
            \]
            This may or may not take on the ``value'' of $\infty$. 
            \item If $f$ takes on values in $\mathbb{R}$, then give the functions $f^+ = \max\{f, 0\}$ and $f^- = \min\{f, 0\}$. 
            Give the Lebesgue integral of $f$ with respect to $\mu$ as:
            \[
                \int_X f(x)\;d\mu(x) = \int_X f^+(x)\;d\mu(x) - \int_X f^-\;d\mu(x),
            \]
            provided that both of the integrals on the right exist and at least one of them is finite. 
            \item Lastly, if $f$ takes on values in $\mathbb{C}$, then let $u$ and $v$ be the functions given by $u = \Re f$ and $v = \Im f$, 
            i.e. $f(x) = u(x) + iv(x)$. Then define the Lebesgue integral of $f$ relative to $\mu$ as:
            \[
                \int_X f(x)\;d\mu(x) = \int_X u(x) \;d\mu(x) + i \int_X v(x)\;d\mu(x).
            \] 
        \end{itemize}
        We usually write $\int f\;d\mu$ or just $\int f$ if the implied parts are easy to understand. Also we write $\int_A f$
        to stand for $\int_X f\mathbbm{1}_A$. In total, these parts define the {\bf Lebesgue integral of $f$ with respect to $\mu$}.  
    \end{definition}
    \begin{theorem}[Cauchy's Integral Formula]
        Let $U$ be an open subset of the $\mathbb{C}$ and suppose that the disk 
        \[
            D(z_0, r) = \{z \in \mathbb{C} \mid |z-z_0| < r\}
        \]
        is completely contained in $U$. Let $f:U\to \mathbb{C}$ be a holomorphic function and let $\gamma$ be the circle oriented
        counterclockwise forming the boundary of $D(z_0, r)$. Then for every interior point $a$ of $D(z_0, r)$, 
        \[
            f(a) = \frac{1}{2\pi i} \oint_\gamma \;\frac{f(z)}{za}\;dz.
        \] 
        Moreover, because $(z-a)^{-1}$ can be given by the power series:
        \[
            \frac{1}{z-a} = \frac{1}{z}\sum_{n=0}^\infty \left(\frac{a}{z}\right)^n
        \]
        it follows that $f$ is analytic, infinitely differentiable, and 
        \[
            f^{(n)}(a) = \frac{n!}{2\pi i} \oint_\gamma \;\frac{f(z)}{(z-a)^{n+1}}\;dz.
        \]
    \end{theorem}
    Here are examples of the major fonts:
    \begin{description}
        \item[Regular] 
        \subitem ABCDEFGHIJKLMNOPQRSTUVWXYZ 
        \subitem abcdefghijklmnopqrstuvwxyz 
        \subitem 1234567890
        \item[Slshape] 
        \subitem {\slshape ABCDEFGHIJKLMNOPQRSTUVWXYZ}
        \subitem {\slshape abcdefghijklmnopqrstuvwxyz}
        \subitem {\slshape 1234567890}
        \item[Bfseries] 
        \subitem {\bfseries ABCDEFGHIJKLMNOPQRSTUVWXYZ}
        \subitem {\bfseries abcdefghijklmnopqrstuvwxyz}
        \subitem {\bfseries 1234567890}
        \item[Slshape] 
        \subitem {\scshape ABCDEFGHIJKLMNOPQRSTUVWXYZ}
        \subitem {\scshape abcdefghijklmnopqrstuvwxyz}
        \subitem {\scshape 1234567890}
        \item[Mathematics] 
        \subitem $ABCDEFGHIJKLMNOPQRSTUVWXYZ$
        \subitem $abcdefghijklmnopqrstuvwxyz$
        \subitem $1234567890$
        \item[Mathcal] 
        \subitem $\mathcal{ABCDEFGHIJKLMNOPQRSTUVWXYZ}$
        \subitem $\mathcal{abcdefghijklmnopqrstuvwxyz}$
        \subitem $\mathcal{1234567890}$
        \item[Mathbb] 
        \subitem $\mathbb{ABCDEFGHIJKLMNOPQRSTUVWXYZ}$
        \subitem $\mathbb{abcdefghijklmnopqrstuvwxyz}$
        \subitem $\mathbb{1234567890}$
        \item[Mathds] 
        \subitem $\mathds{ABCDEFGHIJKLMNOPQRSTUVWXYZ}$
        \subitem $\mathds{abcdefghijklmnopqrstuvwxyz}$
        \subitem $\mathds{1234567890}$
        \item[Mathbbm] 
        \subitem $\mathbbm{ABCDEFGHIJKLMNOPQRSTUVWXYZ}$
        \subitem $\mathbbm{abcdefghijklmnopqrstuvwxyz}$
        \subitem $\mathbbm{1234567890}$
        \item[Mathfrak]
        \subitem $\mathfrak{ABCDEFGHIJKLMNOPQRSTUVWXYZ}$
        \subitem $\mathfrak{abcdefghijklmnopqrstuvwxyz}$
        \subitem $\mathfrak{1234567890}$
        \item[Mathscr]
        \subitem $\mathscr{ABCDEFGHIJKLMNOPQRSTUVWXYZ}$
        \subitem $\mathscr{abcdefghijklmnopqrstuvwxyz}$
        \subitem $\mathscr{1234567890}$
        \item[Greek] $\alpha\beta\gamma\delta\epsilon\varepsilon\zeta\eta\theta\vartheta\iota\kappa\varkappa\lambda\mu\nu\xi\pi\varpi\rho\varrho\sigma\varsigma\tau\upsilon\phi\varphi\chi\phi\varphi\psi\omega$
        \item[Greek (Capital)] $\Gamma\varGamma\Delta\varDelta\Theta\varTheta\Lambda\varLambda\Xi\varXi\Pi\varPi\Sigma\varSigma\Phi\varPhi\Psi\varPsi\Omega\varOmega$
        \item[Common symbols] $\int\oint\iint\oiint\sum\prod\bigcup\bigcap\bigvee\bigwedge\bigoplus\bigotimes$
        \item[Common symbols (displaystyle)] $\displaystyle \int\oint\iint\oiint\sum\prod\bigcup\bigcap\bigvee\bigwedge\bigoplus\bigotimes$
        \item[Inline Sup/Sub-scripts] $\lim_{n\to\infty}\inf_\alpha\sup_{\stackrel{f:\mathbb{R} \to \mathbb{R}}{f \text{ is }1-\text{Lipschitz}}}\sum_{\alpha\in\mathcal{A}}\prod_{n=1}^\infty\int_a^b$
        \item[Inline Sup/Sub-scripts (displaystyle)] $\displaystyle \lim_{n\to\infty}\inf_\alpha\sup_{\stackrel{f:\mathbb{R} \to \mathbb{R}}{f \text{ is }1-\text{Lipschitz}}}\sum_{\alpha\in\mathcal{A}}\prod_{n=1}^\infty\int_a^b$
    \end{description}
    \newpage
\end{document}
